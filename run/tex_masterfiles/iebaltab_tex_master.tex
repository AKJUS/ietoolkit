%%% Table created in Stata by iebaltab
%%% (https://github.com/worldbank/ietoolkit)
%%% (https://dimewiki.worldbank.org/iebaltab)

\documentclass{report}

% ----- Preamble 
\usepackage[utf8]{inputenc}
\usepackage{adjustbox}
\usepackage{threeparttable}
\usepackage[]{geometry}
\geometry{landscape, margin=1in}


% ----- End of preamble 

\begin{document}



% Table 1
\begin{table}
  \centering
  \caption{iebt-tex1}
  \begin{adjustbox}{max width=\textwidth}
    \begin{threeparttable}[!h]
	  %%% Table created in Stata by command iebaltab
%%% (https://github.com/worldbank/ietoolkit)
%%% (https://dimewiki.worldbank.org/iebaltab)
%%% The command was specified exactly like this: 
%%% iebaltab weight price , savecsv("run/iebaltab/outputs/iebt-csv1") savexlsx("run/iebaltab/outputs/iebt-xlsx1") savetex("run/iebaltab/outputs/iebt-tex1") texnotefile("run/iebaltab/outputs/iebt-tex1-note") groupvar(tmt_cl) replace ftest feqtest control(1) cov(mpg) fixed(foreign)

\begin{tabular}{@{\extracolsep{5pt}}lcccccccc}
\\[-1.8ex]\hline \hline \\[-1.8ex]
 & \multicolumn{2}{c}{(1)}  & \multicolumn{2}{c}{(2)}  & \multicolumn{2}{c}{F-test for balance} & \multicolumn{2}{c}{(2)-(1)} \\
 & \multicolumn{2}{c}{1}  & \multicolumn{2}{c}{0}  & \multicolumn{2}{c}{across all groups} & \multicolumn{2}{c}{Pairwise t-test}  \\
Variable & N & Mean/(SE) & N & Mean/(SE) & N & F-stat/P-value & N & Mean difference \\ \hline \\[-1.8ex] 
weight   & 43    & 2972.558    & 31    & 3084.516    & 74    & 2.679    & 74    & 111.958   \\
 &   & (130.959)  &   & (117.753)  &   & 0.106  &   &  \\ [1ex]
price   & 43    & 6312.465    & 31    & 5961.065    & 74    & 0.928    & 74    & -351.401   \\
 &   & (486.238)  &   & (470.476)  &   & 0.339  &   &  \\ [1ex]
\hline \\[-1.8ex]
F-test of joint significance (F-stat) & &   & &   & &     & &  1.325  \\
F-test, number of observations & &   & &   & &   & &  74  \\
\hline \hline \\[-1.8ex]

\end{tabular}

	  \begin{tablenotes}[flushleft]
	    \item\hspace{-.25em}Covariate(s) used in pairwise and f-test regressions: [mpg]. Fixed effect used in pairwise and f-test regressions: [foreign]. Significance: ***=1\%, **=5\%, *=10\%. Full user input as written by user: [iebaltab weight price , savecsv("C:/Users/wb462869/GitHub/ietoolkit/run/output/iebaltab/iebt-csv1") savexlsx("C:/Users/wb462869/GitHub/ietoolkit/run/output/iebaltab/iebt-xlsx1") savetex("C:/Users/wb462869/GitHub/ietoolkit/run/output/iebaltab/iebt-tex1") texnotefile("C:/Users/wb462869/GitHub/ietoolkit/run/output/iebaltab/iebt-tex1-note") grpvar(tmt\_cl) replace ftest feqtest control(1) cov(mpg) fixed(foreign)]

	  \end{tablenotes}
    \end{threeparttable}
  \end{adjustbox}
\end{table}

% Table 2
\begin{table}[!h]
	\centering
	\caption{iebt-tex2}
	\begin{adjustbox}{max width=\textwidth}
		%%% Table created in Stata by command iebaltab
%%% (https://github.com/worldbank/ietoolkit)
%%% (https://dimewiki.worldbank.org/iebaltab)
%%% The command was specified exactly like this: 
%%% iebaltab weight price , groupvar(tmt_cl) replace ftest feqtest control(1) savetex("C:/Users/wb462869/GitHub/ietoolkit/run/output/iebaltab/iebt-tex2") cov(mpg) fixed(foreign)

\begin{tabular}{@{\extracolsep{5pt}}lcccccccc}
\\[-1.8ex]\hline \hline \\[-1.8ex]
 & \multicolumn{2}{c}{(1)}  & \multicolumn{2}{c}{(2)}  & \multicolumn{2}{c}{F-test for balance} & \multicolumn{2}{c}{(2)-(1)} \\
 & \multicolumn{2}{c}{1}  & \multicolumn{2}{c}{0}  & \multicolumn{2}{c}{across all groups} & \multicolumn{2}{c}{Pairwise t-test}  \\
Variable & N & Mean/(SE) & N & Mean/(SE) & N & F-stat/P-value & N & Mean difference \\ \hline \\[-1.8ex] 
weight   & 43    & 2972.558    & 31    & 3084.516    & 74    & 2.679    & 74    & 111.958   \\
 &   & (130.959)  &   & (117.753)  &   & 0.106  &   &  \\ [1ex]
price   & 43    & 6312.465    & 31    & 5961.065    & 74    & 0.928    & 74    & -351.401   \\
 &   & (486.238)  &   & (470.476)  &   & 0.339  &   &  \\ [1ex]
\hline \\[-1.8ex]
F-test of joint significance (F-stat) & &   & &   & &     & &  1.325  \\
F-test, number of observations & &   & &   & &   & &  74  \\
\hline \hline \\[-1.8ex]
%%% This is the note. If it does not have the correct margins, use texnotewidth() option or change the number before '\textwidth' in line below to fit it to table size.
\multicolumn{8}{@{} p{\textwidth}}{If the table includes missing values (.n, .o, .v etc.) see the Missing values section in the help file for the Stata command iebaltab for definitions of these values. Covariate(s) used in pairwise and f-test regressions: [mpg]. Fixed effect used in pairwise and f-test regressions: [foreign]. Significance: ***=.01, **=.05, *=.1. Full user input as written by user: [iebaltab weight price , groupvar(tmt\_cl) replace ftest feqtest control(1) savetex("C:/Users/wb462869/GitHub/ietoolkit/run/output/iebaltab/iebt-tex2") cov(mpg) fixed(foreign)]}

\end{tabular}
		
	\end{adjustbox}
\end{table}

% Table 3
\begin{table}[!h]
	\centering
	\caption{iebt-tex3}
	\begin{adjustbox}{max width=\textwidth}
		%%% Table created in Stata by command iebaltab
%%% (https://github.com/worldbank/ietoolkit)
%%% (https://dimewiki.worldbank.org/iebaltab)
%%% The command was specified exactly like this: 
%%% iebaltab weight price , groupvar(tmt_cl) replace ftest feqtest control(1) savetex("C:/Users/wb462869/github/ietoolkit/run/iebaltab/outputs/iebaltab1/iebt-tex3") cov(mpg) fixed(foreign) texcolwidth(4cm) addnote("Options used: texcolwidth(3cm) short first column ")

\begin{tabular}{@{\extracolsep{5pt}}p{4cm}cccccccc}
\\[-1.8ex]\hline \hline \\[-1.8ex]
 & \multicolumn{2}{c}{(1)}  & \multicolumn{2}{c}{(2)}  & \multicolumn{2}{c}{F-test for balance} & \multicolumn{2}{c}{(2)-(1)} \\
 & \multicolumn{2}{c}{1}  & \multicolumn{2}{c}{0}  & \multicolumn{2}{c}{across all groups} & \multicolumn{2}{c}{Pairwise t-test}  \\
Variable & N & Mean/(SE) & N & Mean/(SE) & N & F-stat/P-value & N & Mean difference \\ \hline \\[-1.8ex] 
weight   & 43    & 2972.558    & 31    & 3084.516    & 74    & 2.679    & 74    & 111.958   \\
 &   & (130.959)  &   & (117.753)  &   & 0.106  &   &  \\ [1ex]
price   & 43    & 6312.465    & 31    & 5961.065    & 74    & 0.928    & 74    & -351.401   \\
 &   & (486.238)  &   & (470.476)  &   & 0.339  &   &  \\ [1ex]
\hline \\[-1.8ex]
F-test of joint significance (F-stat) & &   & &   & &     & &  1.325  \\
F-test, number of observations & &   & &   & &   & &  74  \\
\hline \hline \\[-1.8ex]
%%% This is the note. If it does not have the correct margins, use texnotewidth() option or change the number before '\textwidth' in line below to fit it to table size.
\multicolumn{8}{@{} p{\textwidth}}{If the table includes missing values (.n, .o, .v etc.) see the Missing values section in the help file for the Stata command iebaltab for definitions of these values. Covariate(s) used in pairwise and f-test regressions: [mpg]. Fixed effect used in pairwise and f-test regressions: [foreign]. Significance: ***=.01, **=.05, *=.1. Full user input as written by user: [iebaltab weight price , groupvar(tmt\_cl) replace ftest feqtest control(1) savetex("C:/Users/wb462869/github/ietoolkit/run/iebaltab/outputs/iebaltab1/iebt-tex3") cov(mpg) fixed(foreign) texcolwidth(4cm) addnote("Options used: texcolwidth(3cm) short first column ")] Options used: texcolwidth(3cm) short first column }

\end{tabular}
		
	\end{adjustbox}
\end{table}

Table 4 is stand alone doc

Table 5 is stand alone doc

% Table 6
\begin{table}
	\centering
	\caption{File iebt-tex6}
	\begin{adjustbox}{max width=\textwidth}
		\begin{threeparttable}[!h]
			%%% Table created in Stata by command iebaltab
%%% (https://github.com/worldbank/ietoolkit)
%%% (https://dimewiki.worldbank.org/iebaltab)
%%% The command was specified exactly like this: 
%%% iebaltab weight price , savecsv("C:/Users/wb462869/GitHub/ietoolkit/run/output/iebaltab/iebt-csv6") savexlsx("C:/Users/wb462869/GitHub/ietoolkit/run/output/iebaltab/iebt-xlsx6") savetex("C:/Users/wb462869/GitHub/ietoolkit/run/output/iebaltab/iebt-tex6") texnotefile("C:/Users/wb462869/GitHub/ietoolkit/run/output/iebaltab/iebt-tex6-note") grpvar(tmt) replace ftest feqtest total rowvarlabels cov(mpg) fixed(foreign) tbladdnote("Many groups, rowvarlabels")

\begin{tabular}{@{\extracolsep{5pt}}lcccccccccccccccccccccccc}
\\[-1.8ex]\hline \hline \\[-1.8ex]
 & \multicolumn{2}{c}{(1)}  & \multicolumn{2}{c}{(2)}  & \multicolumn{2}{c}{(3)}  & \multicolumn{2}{c}{(4)}  & \multicolumn{2}{c}{(5)}  & \multicolumn{2}{c}{F-test for balance} & \multicolumn{2}{c}{(2)-(3)} & \multicolumn{2}{c}{(2)-(4)} & \multicolumn{2}{c}{(2)-(5)} & \multicolumn{2}{c}{(3)-(4)} & \multicolumn{2}{c}{(3)-(5)} & \multicolumn{2}{c}{(4)-(5)} \\
 & \multicolumn{2}{c}{Total}  & \multicolumn{2}{c}{Oi in \%}  & \multicolumn{2}{c}{control/unobserved}  & \multicolumn{2}{c}{taco \& salsa}  & \multicolumn{2}{c}{10231}  & \multicolumn{2}{c}{across all groups} & \multicolumn{12}{c}{Pairwise t-test}  \\
Variable & N & Mean/(SE) & N & Mean/(SE) & N & Mean/(SE) & N & Mean/(SE) & N & Mean/(SE) & N & F-stat/P-value & N & Mean difference & N & Mean difference & N & Mean difference & N & Mean difference & N & Mean difference & N & Mean difference \\ \hline \\[-1.8ex] 
Weight (lbs.)   & 74    & 3019.459    & 22    & 3176.818    & 21    & 2758.571    & 19    & 3004.211    & 12    & 3211.667    & 74    & 1.936    & 43    & 418.247    & 41    & 172.608    & 34    & -34.848    & 40    & -245.639    & 33    & -453.095    & 31    & -207.456   \\
 &   & (90.347)  &   & (172.374)  &   & (191.277)  &   & (157.068)  &   & (177.127)  &   & 0.132  &   &  &   &  &   &  &   &  &   &  &   &   \\
Price   & 74    & 6165.257    & 22    & 7209.182    & 21    & 5373.048    & 19    & 5858.526    & 12    & 6123.417    & 74    & 1.101    & 43    & 1836.134    & 41    & 1350.656    & 34    & 1085.765    & 40    & -485.479    & 33    & -750.369    & 31    & -264.890   \\
 &   & (342.872)  &   & (776.498)  &   & (516.864)  &   & (605.744)  &   & (777.763)  &   & 0.355  &   &  &   &  &   &  &   &  &   &  &   &   \\
\hline \\[-1.8ex]
F-test of joint significance (F-stat) & &   & &   & &   & &   & &   & &     & &  1.388    & &  2.807    & &  0.828    & &  0.605    & &  0.012    & &  0.182   \\
F-test, number of observations & &   & &   & &   & &   & &   & &   & &  43  & &  41  & &  34  & &  40  & &  33  & &  31   \\
\hline \\[-1.8ex]

\end{tabular}

			\begin{tablenotes}[flushleft]
				\item\hspace{-.25em}If the table includes missing values (.n, .o, .v etc.) see the Missing values section in the help file for the Stata command iebaltab for definitions of these values. Covariate(s) used in pairwise and f-test regressions: [mpg]. Fixed effect used in pairwise and f-test regressions: [foreign]. Significance: ***=.01, **=.05, *=.1. Full user input as written by user: [iebaltab weight price , savecsv("C:/Users/wb462869/GitHub/ietoolkit/run/output/iebaltab/iebt-csv6") savexlsx("C:/Users/wb462869/GitHub/ietoolkit/run/output/iebaltab/iebt-xlsx6") savetex("C:/Users/wb462869/GitHub/ietoolkit/run/output/iebaltab/iebt-tex6") texnotefile("C:/Users/wb462869/GitHub/ietoolkit/run/output/iebaltab/iebt-tex6-note") groupvar(tmt) replace ftest feqtest total rowvarlabels cov(mpg) fixed(foreign) noteappend("Many groups, rowvarlabels")] Many groups, rowvarlabels

			\end{tablenotes}
		\end{threeparttable}
	\end{adjustbox}
\end{table}

% Table 7
\begin{table}
	\centering
	\caption{File iebt-tex7}
	\begin{adjustbox}{max width=\textwidth}
		\begin{threeparttable}[!h]
			%%% Table created in Stata by command iebaltab
%%% (https://github.com/worldbank/ietoolkit)
%%% (https://dimewiki.worldbank.org/iebaltab)
%%% The command was specified exactly like this: 
%%% iebaltab weight price , savecsv("C:/Users/wb462869/GitHub/ietoolkit/run/output/iebaltab/iebt-csv7") savexlsx("C:/Users/wb462869/GitHub/ietoolkit/run/output/iebaltab/iebt-xlsx7") savetex("C:/Users/wb462869/GitHub/ietoolkit/run/output/iebaltab/iebt-tex7") texnotefile("C:/Users/wb462869/GitHub/ietoolkit/run/output/iebaltab/iebt-tex7-note") grpvar(tmt) replace total order(4 10231) control(6) grpcodes cov(mpg) fixed(foreign) tbladdnote("column order should be 4 10231 6 2, and 6 is control so pair test only with this group")

\begin{tabular}{@{\extracolsep{5pt}}lcccccccccccccccc}
\\[-1.8ex]\hline \hline \\[-1.8ex]
 & \multicolumn{2}{c}{(1)}  & \multicolumn{2}{c}{(2)}  & \multicolumn{2}{c}{(3)}  & \multicolumn{2}{c}{(4)}  & \multicolumn{2}{c}{(5)}  & \multicolumn{2}{c}{(1)-(3)} & \multicolumn{2}{c}{(2)-(3)} & \multicolumn{2}{c}{(4)-(3)} \\
 & \multicolumn{2}{c}{4}  & \multicolumn{2}{c}{10231}  & \multicolumn{2}{c}{6}  & \multicolumn{2}{c}{2}  & \multicolumn{2}{c}{Total}  & \multicolumn{6}{c}{Pairwise t-test}  \\
Variable & N & Mean/(SE) & N & Mean/(SE) & N & Mean/(SE) & N & Mean/(SE) & N & Mean/(SE) & N & Mean difference & N & Mean difference & N & Mean difference \\ \hline \\[-1.8ex] 
weight   & 21    &  2758.571    & 12    &  3211.667    & 19    &  3004.211    & 22    &  3176.818    & 74    &  3019.459    & 40    &  -245.639    & 31    &   207.456    & 41    &   172.608   \\
 &   &   191.277  &   &   177.127  &   &   157.068  &   &   172.374  &   &    90.347  &   &  &   &  &   &   \\
price   & 21    &  5373.048    & 12    &  6123.417    & 19    &  5858.526    & 22    &  7209.182    & 74    &  6165.257    & 40    &  -485.479    & 31    &   264.890    & 41    &  1350.656   \\
 &   &   516.864  &   &   777.763  &   &   605.744  &   &   776.498  &   &   342.872  &   &  &   &  &   &   \\
\hline \\[-1.8ex]

\end{tabular}

			\begin{tablenotes}[flushleft]
				\item\hspace{-.25em}If the table includes missing values (.n, .o, .v etc.) see the Missing values section in the help file for the Stata command iebaltab for definitions of these values. Covariate(s) used in pairwise regressions: [mpg]. Fixed effect used in pairwise regressions: [foreign]. Significance: ***=.01, **=.05, *=.1. Full user input as written by user: [iebaltab weight price , savecsv("C:/Users/wb462869/GitHub/ietoolkit/run/output/iebaltab/iebt-csv7") savexlsx("C:/Users/wb462869/GitHub/ietoolkit/run/output/iebaltab/iebt-xlsx7") savetex("C:/Users/wb462869/GitHub/ietoolkit/run/output/iebaltab/iebt-tex7") texnotefile("C:/Users/wb462869/GitHub/ietoolkit/run/output/iebaltab/iebt-tex7-note") groupvar(tmt) replace total order(4 10231) control(6) groupcodes cov(mpg) fixed(foreign) tableaddnote("column order should be 4 10231 6 2, and 6 is control so pair test only with this group")] column order should be 4 10231 6 2, and 6 is control so pair test only with this group

			\end{tablenotes}
		\end{threeparttable}
	\end{adjustbox}
\end{table}

% Table 8
\begin{table}
	\centering
	\caption{File iebt-tex8}
	\begin{adjustbox}{max width=\textwidth}
		\begin{threeparttable}[!h]
			%%% Table created in Stata by command iebaltab
%%% (https://github.com/worldbank/ietoolkit)
%%% (https://dimewiki.worldbank.org/iebaltab)
%%% The command was specified exactly like this: 
%%% iebaltab weight price headroom , savecsv("C:/Users/wb462869/GitHub/ietoolkit/run/output/iebaltab/iebt-csv8") savexlsx("C:/Users/wb462869/GitHub/ietoolkit/run/output/iebaltab/iebt-xlsx8") savetex("C:/Users/wb462869/GitHub/ietoolkit/run/output/iebaltab/iebt-tex8") texnotefile("C:/Users/wb462869/GitHub/ietoolkit/run/output/iebaltab/iebt-tex8-note") grpvar(tmt) replace total control(6) rowvarlabels grplabels(`"6 Quotes, and comma "," @ 4 Pizza & Pineapple (USD$) "') totallabel("Total single ' quote") cov(mpg) fixed(foreign) rowlabels(`"price St*r and sub _script @ headroom Headroom "Height" quote "') tbladdnote("Row column manual ($) lables.")

\begin{tabular}{@{\extracolsep{5pt}}lcccccccccccccccc}
\\[-1.8ex]\hline \hline \\[-1.8ex]
 & \multicolumn{2}{c}{(1)}  & \multicolumn{2}{c}{(2)}  & \multicolumn{2}{c}{(3)}  & \multicolumn{2}{c}{(4)}  & \multicolumn{2}{c}{(5)}  & \multicolumn{2}{c}{(3)-(2)} & \multicolumn{2}{c}{(4)-(2)} & \multicolumn{2}{c}{(5)-(2)} \\
 & \multicolumn{2}{c}{Total single ' quote}  & \multicolumn{2}{c}{Quotes, and comma ","}  & \multicolumn{2}{c}{Oi in \%}  & \multicolumn{2}{c}{Pizza \& Pineapple (USD\$)}  & \multicolumn{2}{c}{10231}  & \multicolumn{6}{c}{Pairwise t-test}  \\
Variable & N & Mean/(SE) & N & Mean/(SE) & N & Mean/(SE) & N & Mean/(SE) & N & Mean/(SE) & N & Mean difference & N & Mean difference & N & Mean difference \\ \hline \\[-1.8ex] 
Weight (USD\$)   & 74    &  3019.459    & 19    &  3004.211    & 22    &  3176.818    & 21    &  2758.571    & 12    &  3211.667    & 41    &   172.608    & 40    &  -245.639    & 31    &   207.456   \\
 &   &    90.347  &   &   157.068  &   &   172.374  &   &   191.277  &   &   177.127  &   &  &   &  &   &   \\
St*r and sub \_script   & 74    &  6165.257    & 19    &  5858.526    & 22    &  7209.182    & 21    &  5373.048    & 12    &  6123.417    & 41    &  1350.656    & 40    &  -485.479    & 31    &   264.890   \\
 &   &   342.872  &   &   605.744  &   &   776.498  &   &   516.864  &   &   777.763  &   &  &   &  &   &   \\
Headroom "Height" quote   & 74    &     2.993    & 19    &     2.895    & 22    &     3.114    & 21    &     2.952    & 12    &     3.000    & 41    &     0.219    & 40    &     0.058    & 31    &     0.105   \\
 &   &     0.098  &   &     0.190  &   &     0.189  &   &     0.176  &   &     0.275  &   &  &   &  &   &   \\
\hline \\[-1.8ex]

\end{tabular}

			\begin{tablenotes}[flushleft]
				\item\hspace{-.25em}If the table includes missing values (.n, .o, .v etc.) see the Missing values section in the help file for the Stata command iebaltab for definitions of these values. Covariate(s) used in pairwise regressions: [mpg]. Fixed effect used in pairwise regressions: [foreign]. Significance: ***=.01, **=.05, *=.1. Full user input as written by user: [iebaltab weight price headroom , savecsv("run/iebaltab/outputs/iebt-csv8") savexlsx("run/iebaltab/outputs/iebt-xlsx8") savetex("run/iebaltab/outputs/iebt-tex8") texnotefile("run/iebaltab/outputs/iebt-tex8-note") groupvar(tmt) replace total control(6) rowvarlabels grouplabels(`"6 Quotes, and comma "," @ 4 Pizza \& Pineapple (USD\$) "') totallabel("Total single ' quote") cov(mpg) fixed(foreign) rowlabels(`"price St*r and sub \_script @ headroom Headroom "Height" quote "') addnote("Row column manual (\$) lables.")] Row column manual (\$) lables.

			\end{tablenotes}
		\end{threeparttable}
	\end{adjustbox}
\end{table}


% Table 9
\begin{table}
	\centering
	\caption{File iebt-tex9}
	\begin{adjustbox}{max width=\textwidth}
		\begin{threeparttable}[!h]
			%%% Table created in Stata by command iebaltab
%%% (https://github.com/worldbank/ietoolkit)
%%% (https://dimewiki.worldbank.org/iebaltab)
%%% The command was specified exactly like this: 
%%% iebaltab weight price , savecsv("C:/Users/wb462869/GitHub/ietoolkit/run/output/iebaltab/iebt-csv9") savexlsx("C:/Users/wb462869/GitHub/ietoolkit/run/output/iebaltab/iebt-xlsx9") savetex("C:/Users/wb462869/GitHub/ietoolkit/run/output/iebaltab/iebt-tex9") texnotefile("C:/Users/wb462869/GitHub/ietoolkit/run/output/iebaltab/iebt-tex9-note") grpvar(tmt_cl) replace ftest feqtest cov(mpg) fixed(foreign) tbladdnote("Warning for missing value in fixedeffect(foreign) and in covariates(mpg)")

\begin{tabular}{@{\extracolsep{5pt}}lcccccccc}
\\[-1.8ex]\hline \hline \\[-1.8ex]
 & \multicolumn{2}{c}{(1)}  & \multicolumn{2}{c}{(2)}  & \multicolumn{2}{c}{F-test for balance} & \multicolumn{2}{c}{(1)-(2)} \\
 & \multicolumn{2}{c}{0}  & \multicolumn{2}{c}{1}  & \multicolumn{2}{c}{across all groups} & \multicolumn{2}{c}{Pairwise t-test}  \\
Variable & N & Mean/(SE) & N & Mean/(SE) & N & F-stat/P-value & N & Mean difference \\ \hline \\[-1.8ex] 
weight   & 31    &  3084.516    & 42    &  2980.476    & 71    &     2.681    & 71    &   104.040   \\
 &   &   117.753  &   &   133.869  &   &     0.106  &   &   \\
price   & 31    &  5961.065    & 43    &  6312.465    & 72    &     0.742    & 72    &  -351.401   \\
 &   &   470.476  &   &   486.238  &   &     0.392  &   &   \\
\hline \\[-1.8ex]
F-test of joint significance (F-stat) & &   & &   & &     & &      1.324   \\
F-test, number of observations & &   & &   & &   & &  71   \\
\hline \\[-1.8ex]

\end{tabular}

			\begin{tablenotes}[flushleft]
				\item\hspace{-.25em}If the table includes missing values (.n, .o, .v etc.) see the Missing values section in the help file for the Stata command iebaltab for definitions of these values. Covariate(s) used in pairwise and f-test regressions: [mpg]. Fixed effect used in pairwise and f-test regressions: [foreign]. Significance: ***=.01, **=.05, *=.1. Full user input as written by user: [iebaltab weight price , savecsv("C:/Users/wb462869/GitHub/ietoolkit/run/output/iebaltab/iebt-csv9") savexlsx("C:/Users/wb462869/GitHub/ietoolkit/run/output/iebaltab/iebt-xlsx9") savetex("C:/Users/wb462869/GitHub/ietoolkit/run/output/iebaltab/iebt-tex9") texnotefile("C:/Users/wb462869/GitHub/ietoolkit/run/output/iebaltab/iebt-tex9-note") groupvar(tmt\_cl) replace ftest feqtest cov(mpg) fixed(foreign) noteappend("Warning for missing value in fixedeffect(foreign) and in covariates(mpg)")] Warning for missing value in fixedeffect(foreign) and in covariates(mpg)

			\end{tablenotes}
		\end{threeparttable}
	\end{adjustbox}
\end{table}

% Table 10
\begin{table}
	\centering
	\caption{File iebt-tex10}
	\begin{adjustbox}{max width=\textwidth}
		\begin{threeparttable}[!h]
			%%% Table created in Stata by command iebaltab
%%% (https://github.com/worldbank/ietoolkit)
%%% (https://dimewiki.worldbank.org/iebaltab)
%%% The command was specified exactly like this: 
%%% iebaltab weight price , savecsv("C:/Users/wb462869/GitHub/ietoolkit/run/output/iebaltab/iebt-csv10") savexlsx("C:/Users/wb462869/GitHub/ietoolkit/run/output/iebaltab/iebt-xlsx10") savetex("C:/Users/wb462869/GitHub/ietoolkit/run/output/iebaltab/iebt-tex10") texnotefile("C:/Users/wb462869/GitHub/ietoolkit/run/output/iebaltab/iebt-tex10-note") groupvar(tmt_cl) replace vce(cluster test_cluster_var) ftest feqtest total cov(mpg) fixed(foreign) tableaddnote("added cluster")

\begin{tabular}{@{\extracolsep{5pt}}lcccccccccc}
\\[-1.8ex]\hline \hline \\[-1.8ex]
 & \multicolumn{2}{c}{(1)}  & \multicolumn{2}{c}{(2)}  & \multicolumn{2}{c}{(3)}  & \multicolumn{2}{c}{F-test for balance} & \multicolumn{2}{c}{(2)-(3)} \\
 & \multicolumn{2}{c}{Total}  & \multicolumn{2}{c}{0}  & \multicolumn{2}{c}{1}  & \multicolumn{2}{c}{across all groups} & \multicolumn{2}{c}{Pairwise t-test}  \\
Variable & N/Clusters & Mean/(SE) & N/Clusters & Mean/(SE) & N/Clusters & Mean/(SE) & N/Clusters & F-stat/P-value & N/Clusters & Mean difference \\ \hline \\[-1.8ex] 
weight   & 74    & 3019.459    & 31    & 3084.516    & 43    & 2972.558    & 74    & 3.353    & 74    & 111.958   \\
 & 4  & (107.435)  & 2  & (98.439)  & 2  & (209.010)  & 4  & 0.164  & 4  &  \\ [1ex]
price   & 74    & 6165.257    & 31    & 5961.065    & 43    & 6312.465    & 74    & 1.695    & 74    & -351.401   \\
 & 4  & (451.828)  & 2  & (125.692)  & 2  & (917.571)  & 4  & 0.284  & 4  &  \\ [1ex]
\hline \\[-1.8ex]
F-test of joint significance (F-stat) & &   & &   & &   & &     & &  1.128  \\
F-test, number of observations & &   & &   & &   & &   & &  74  \\
F-test, number of clusters & &   & &   & &   & &     & &  4  \\
\hline \hline \\[-1.8ex]

\end{tabular}

			\begin{tablenotes}[flushleft]
				\item\hspace{-.25em}If the table includes missing values (.n, .o, .v etc.) see the Missing values section in the help file for the Stata command iebaltab for definitions of these values. Covariate(s) used in pairwise and f-test regressions: [mpg]. Fixed effect used in pairwise and f-test regressions: [foreign]. Significance: ***=.01, **=.05, *=.1. Errors are clustered at variable: [test\_cluster\_var]. Full user input as written by user: [iebaltab weight price , savecsv("C:/Users/wb462869/GitHub/ietoolkit/run/output/iebaltab/iebt-csv10") savexlsx("C:/Users/wb462869/GitHub/ietoolkit/run/output/iebaltab/iebt-xlsx10") savetex("C:/Users/wb462869/GitHub/ietoolkit/run/output/iebaltab/iebt-tex10") texnotefile("C:/Users/wb462869/GitHub/ietoolkit/run/output/iebaltab/iebt-tex10-note") groupvar(tmt\_cl) replace vce(cluster test\_cluster\_var) ftest feqtest total cov(mpg) fixed(foreign) addnote("added cluster")] added cluster

			\end{tablenotes}
		\end{threeparttable}
	\end{adjustbox}
\end{table}

% Table 11
\begin{table}
	\centering
	\caption{File iebt-tex11}
	\begin{adjustbox}{max width=\textwidth}
		\begin{threeparttable}[!h]
			%%% Table created in Stata by command iebaltab
%%% (https://github.com/worldbank/ietoolkit)
%%% (https://dimewiki.worldbank.org/iebaltab)
%%% The command was specified exactly like this: 
%%% iebaltab weight price , savecsv("run/iebaltab/outputs/iebaltab1/iebt-csv11") savexlsx("run/iebaltab/outputs/iebaltab1/iebt-xlsx11") savetex("run/iebaltab/outputs/iebaltab1/iebt-tex11") texnotefile("run/iebaltab/outputs/iebaltab1/iebt-tex11-note") groupvar(tmt_cl) replace onerow cov(mpg) fixed(foreign) addnote("added onerow")

\begin{tabular}{@{\extracolsep{5pt}}lccc}
\\[-1.8ex]\hline \hline \\[-1.8ex]
 & \multicolumn{1}{c}{(1)}  & \multicolumn{1}{c}{(2)}  & \multicolumn{1}{c}{(1)-(2)} \\
 & \multicolumn{1}{c}{0}  & \multicolumn{1}{c}{1}  & \multicolumn{1}{c}{Pairwise t-test}  \\
Variable & Mean/(SE) & Mean/(SE) & Mean difference \\ \hline \\[-1.8ex] 
weight   & 3084.516    & 2972.558    & 111.958   \\
 & (117.753)  & (130.959)  &  \\ [1ex]
price   & 5961.065    & 6312.465    & -351.401   \\
 & (470.476)  & (486.238)  &  \\ [1ex]
\hline \\[-1.8ex]
Number of observations  & 31   & 43  & 74  \\
\hline \hline \\[-1.8ex]

\end{tabular}

			\begin{tablenotes}[flushleft]
				\item\hspace{-.25em}If the table includes missing values (.n, .o, .v etc.) see the Missing values section in the help file for the Stata command iebaltab for definitions of these values. Covariate(s) used in pairwise regressions: [mpg]. Fixed effect used in pairwise regressions: [foreign]. Significance: ***=.01, **=.05, *=.1. Full user input as written by user: [iebaltab weight price , savecsv("run/iebaltab/outputs/iebaltab1/iebt-csv11") savexlsx("run/iebaltab/outputs/iebaltab1/iebt-xlsx11") savetex("run/iebaltab/outputs/iebaltab1/iebt-tex11") texnotefile("run/iebaltab/outputs/iebaltab1/iebt-tex11-note") groupvar(tmt\_cl) replace onerow cov(mpg) fixed(foreign) addnote("added onerow")] added onerow

			\end{tablenotes}
		\end{threeparttable}
	\end{adjustbox}
\end{table}

% Table 12
\begin{table}
	\centering
	\caption{File iebt-tex12}
	\begin{adjustbox}{max width=\textwidth}
		\begin{threeparttable}[!h]
			%%% Table created in Stata by command iebaltab
%%% (https://github.com/worldbank/ietoolkit)
%%% (https://dimewiki.worldbank.org/iebaltab)
%%% The command was specified exactly like this: 
%%% iebaltab weight price , savecsv("run/iebaltab/outputs/iebaltab1/iebt-csv12") savexlsx("run/iebaltab/outputs/iebaltab1/iebt-xlsx12") savetex("run/iebaltab/outputs/iebaltab1/iebt-tex12") texnotefile("run/iebaltab/outputs/iebaltab1/iebt-tex12-note") groupvar(tmt_cl) replace onerow vce(cluster test_cluster_var) ftest feqtest total cov(mpg) fixed(foreign) addnote("added onerow and cluster") browse

\begin{tabular}{@{\extracolsep{5pt}}lccccc}
\\[-1.8ex]\hline \hline \\[-1.8ex]
 & \multicolumn{1}{c}{(1)}  & \multicolumn{1}{c}{(2)}  & \multicolumn{1}{c}{(3)}  & \multicolumn{1}{c}{F-test for balance} & \multicolumn{1}{c}{(2)-(3)} \\
 & \multicolumn{1}{c}{Total}  & \multicolumn{1}{c}{0}  & \multicolumn{1}{c}{1}  & \multicolumn{1}{c}{across all groups} & \multicolumn{1}{c}{Pairwise t-test}  \\
Variable & Mean/(SE) & Mean/(SE) & Mean/(SE) & F-stat/P-value & Mean difference \\ \hline \\[-1.8ex] 
weight   & 3019.459    & 3084.516    & 2972.558    & 3.353    & 111.958   \\
 & (107.435)  & (98.439)  & (209.010)  & 0.164  &  \\ [1ex]
price   & 6165.257    & 5961.065    & 6312.465    & 1.695    & -351.401   \\
 & (451.828)  & (125.692)  & (917.571)  & 0.284  &  \\ [1ex]
\hline \\[-1.8ex]
F-test of joint significance (F-stat) &   &   &   &     &  1.128  \\
\hline \\[-1.8ex]
Number of observations  & 74   & 31   & 43  & 74 & 74  \\
Number of clusters & 4  & 2  & 2  & 4 & 4  \\
\hline \hline \\[-1.8ex]

\end{tabular}

			\begin{tablenotes}[flushleft]
				\item\hspace{-.25em}If the table includes missing values (.n, .o, .v etc.) see the Missing values section in the help file for the Stata command iebaltab for definitions of these values. Covariate(s) used in pairwise and f-test regressions: [mpg]. Fixed effect used in pairwise and f-test regressions: [foreign]. Significance: ***=.01, **=.05, *=.1. Errors are clustered at variable: [test\_cluster\_var]. Full user input as written by user: [iebaltab weight price , savecsv("run/iebaltab/outputs/iebt-csv12") savexlsx("run/iebaltab/outputs/iebt-xlsx12") savetex("run/iebaltab/outputs/iebt-tex12") texnotefile("run/iebaltab/outputs/iebt-tex12-note") groupvar(tmt\_cl) replace onerow vce(cluster test\_cluster\_var) ftest feqtest total cov(mpg) fixed(foreign) addnote("added onerow and cluster") browse] added onerow and cluster

			\end{tablenotes}
		\end{threeparttable}
	\end{adjustbox}
\end{table}

\end{document}